This paper intends to explore the question "How much can I expect to save by adopting technology X?". Products intended to help consumers save on their energy bills have been around for years, ranging from such basics as door and window seals, to more efficient appliances, to modifications requiring major renovation such as improved insulation and windows. However, as technology marches onward, consumers are faced with many more high-tech offerings than in the past. These new options for consumers may impress or intimidate, and too many consumers have been swindled into purchasing technologies they may not use or may gain no benefit from \cite{obrien_think_nodate}.

This paper intends to explore three maturing technologies: solar panels, home battery systems, and smart appliances. These three technologies were chosen because they represent three technologies commonly noticed and discussed by consumers. They are some of the more forward-facing technology options marketed at residential users. Some, such as solar panels, are hard to miss when they are installed on your neighbor's home. Others, such as a home battery or smart appliances, may not be as noticeable but are heavily marketed and discussed in news-media.

This paper will choose to ignore price of installing and maintaining technologies. Rather, this paper will present potential energy savings of combinations of these technologies, leaving it to consumers to make informed decisions on which technologies are right for them.

In order to provide accurate estimates of energy consumption for interested consumers, several sources of real-world data are used. Solar radiation measurements and weather information local to Logan, UT \cite{noauthor_utah_nodate} are used for estimating solar radiation and thus solar panel output. Published time-of-day pricing information is used to determine what consumers can expect to pay for energy in units of dollars/kWh. Additional research on energy usage patterns are taken to represent typical and ideal energy-curves of consumers. A smart battery is simulated using Monte Carlo tree search to decide what action a battery should make to best take advantage of available resources.
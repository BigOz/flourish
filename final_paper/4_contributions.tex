While many of the previously mentioned works relate to what this paper attempts to accomplish, none tie the components together in the same way. Our contributions are as follows:

\begin{itemize}
  \item Similar to other papers, this paper relies on a neural network to predict solar radiation output on different dates and at different times throughout the year. However, this paper is using the results of those predictions as a resource allocated to a smart battery, allowing the battery to run simulations based on real predictions. Thus, the neural network is more of a tool to enable more large-scale predictions, rather than the main focus of research.
  \item While there is existing research on using Monte Carlo tree search to schedule energy consumption to off-peak times, this paper takes a different approach by leaving loads as they are and using Monte Carlo tree search to instead control the operations of a home battery system. This approach was taken to decouple the battery from the rest of the systems in the home. A battery home can have solar panels, or it may not, and the battery should still be able to contribute as meaningfully as possible either way. Having a battery determine when to charge using Monte Carlo tree search makes sense for something such as an electric vehicle, which would require a significant energy draw from a home's electrical system that may not be easily accounted for by a home battery alone. The two technologies using Monte Carlo tree search -- a home battery and an electric vehicle charger -- could possibly exist in the same system.
  \item This paper uses two energy curves to represent those from a smart home and those from a typical, non-smart home. While research has been done into what those power curves are \cite{fischer_we_2014}, nothing could be found concerning using those power curves to understand their relationships with solar power and/or home battery systems, as explored in this paper.
  \item This paper attempts to understand the relationships between multiple technologies. While research exists documenting the advantages of technologies in isolation, nothing could be found demonstrating how the technologies used together might be better than the sum of their individual contributions.
\end{itemize}
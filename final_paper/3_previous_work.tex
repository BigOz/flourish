Significant research has been done in individual areas related to this project.

Weather has long been an explored application for neural networks \cite{baboo_efficient_2010}. Solar radiation prediction using weather conditions and neural networks has been explored \cite{rehman_artificial_2008} \cite{behrang_potential_2010} \cite{soares_modeling_2004}. Most of this research seems to focus on environmental factors, including temperature estimation and agriculture. There has been other exploration of using neural networks to predict solar power generation \cite{chen_online_2011}. This research was specifically catered to large-scale solar power generators and commodities traders.

There has been extensive research into battery systems over the past several years. Some research focuses more on how different charging algorithms impact long-term battery life \cite{bila_grid_2016}. Others provide coverage of evolution of home-storage batteries and where the future might go \cite{restrepo_residential_2015}.

One study \cite{truong_economics_2016} focuses specifically on the Tesla Powerwall battery, the same battery simulated in this paper, as it is used with solar panels in Germany. Though not entirely similar to the studies proposed in this paper, their study did find that coupling a battery with solar panels can be financially beneficial in certain circumstance, similar to findings presented in this paper. One note is that the batteries used in the mentioned study were significantly discounted over the typical retail price.

Single-player Monte Carlo tree search has been explored \cite{schadd_single-player_2012} and found to maintain key advantages over other optimization algorithms such as a* in certain situations. Researchers have also found success using Monte Carlo tree search in planning power usage for off-peak times \cite{golpayegani_collaborative_2015}. One major difference, however, is that the mentioned study focuses mostly on a single consuming appliance -- charging electric vehicles. Our study focuses on Monte Carlo tree search from the point of view of the home battery, not just a single consuming appliance connected to the home.

Modifying energy curves through smart home technologies has been esplored \cite{barker_smartcap:_2012}. The approach in this paper of using published energy curves is simpler than in other studies, but it represents very similar results. There has also been significant research into automated technologies to control commercial buildings \cite{weng_buildingdepot_2013}. In addition, there has been research on having energy loads dynamically adjust to the available energy \cite{verma_brownmap:_2010}. This might be worth considering for future smart-home implementations, but is beyond the scope of this paper.
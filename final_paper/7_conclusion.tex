There can be several conclusions drawn from the results of this study.

\begin{itemize}
    \item "Free" energy will always beat out "cheap" energy. In this study, solar panels represent "free" energy, whose cost is nothing. While the battery can take advantage of "cheap" energy through time-of-day pricing, the costs savings are so small, roughly \$0.015 per kWh over the normal non time-of-day pricing, that it takes lots of careful maneuvers to experience any real relevant savings, particularly enough to justify investing in an expensive technology.
    \item While a standalone battery might be impractical in the scenario laid out in this simulation, a battery system coupled with solar panels might be worth the investment, depending on the circumstances. This is similar to the results from other studies \cite{truong_economics_2016}.
    \item In the smart home scenario outlined in this paper, where loads are merely shifted rather then reduced, smart home technology does not seem to significantly reduce energy costs. Thus, smart home technologies may be better suited to increase comfort and convenience rather than reduce energy bills. However, smart home technologies exist that can actually reduce overall energy usage. Such technologies may be worth studying in a formal way.
    \item There are many factors that play into this study that could vary drastically from one situation to another. This paper made many assumptions on load, solar panel size, battery size, usage patterns, and pricing. If this study were to be conducted using different information representing a different circumstance, locale, etc, the results might be very different. For instance, in an area with drastic time-of-day differences in pricing, a standalone home battery system might be found to be more relevant. Other areas with reduced solar radiation might show solar panels to be less of an advantage. Different energy usage curves might show smart home technology to be more beneficial than this study showed.
\end{itemize}


\boldmath Once an ideal saved for research papers and environmentalists, conservation of energy is now a mainstream topic, frequently discussed in news-media, classrooms, and water cooler conversations. As typically follows popular trends, consumers are barraged with a host of technologies aimed to reduce one's carbon footprint while simultaneously lightening one's wallet. While telemarketers, advertisements, even door-to-door salesmen approach, consumers may understandably feel overwhelmed and under-informed at the options. This paper analyzes three of the most popular options marketed at residential consumers: solar panels, home battery systems, and smart appliances or "smart home" technologies. Using a combination of Monte Carlo tree search, neural networks, energy archetypes or usage profiles, and published time-of-day energy pricing information, the entirety of year 2016 is simulated, suggesting what consumers with various technologies may have spent on energy bills that year. The results suggest that certain technologies, specifically solar panels, may provide most of the benefit of current popular technologies. Thoughts on what is holding back other technologies, as well as potential areas for future research, follows.